\setlength{\headheight}{19.53pt}
\setlength{\headsep}{1.8em}
\setlength{\belowcaptionskip}{-12pt}
\addbibresource{sources.bib}
\setminted{fontsize=\footnotesize}
\AfterEndEnvironment{minted}{\vspace*{-0.8cm}}
\tikzexternalize[prefix=tikz/]
\renewcommand{\operatorcolor}{black}
% \tikzset{>={Latex[width=3mm,length=3mm]}}
\usetikzlibrary{arrows.meta,calc,decorations.pathreplacing,graphs,quotes}

\newcommand{\iarronly}[1]{
  \node [currarrow, color=red, anchor=center,
    rotate=\ctikzgetdirection{#1-Iarrow}] at (#1-Ipos) {};
}
\newcommand{\varronly}[1]{
  \draw [color=blue] (#1-Vfrom) .. controls (#1-Vcont1)
  and (#1-Vcont2).. (#1-Vto) node [currarrow,
      sloped, anchor=tip, allow upside down,pos=1]{};
}
\NewDocumentCommand{\fixedvlen}{O{0.7cm} m m O{}}{
  % [semilength]{node}{label}[extra options]
  % get the center of the standard arrow
  \coordinate (#2-Vcenter) at ($(#2-Vfrom)!0.5!(#2-Vto)$);
  % draw an arrow of a fixed size around that center and on the same line
  \draw[-Triangle, #4] ($(#2-Vcenter)!#1!(#2-Vfrom)$) -- ($(#2-Vcenter)!#1!(#2-Vto)$);
  % position the label as in the normal voltages
  \node[anchor=\ctikzgetanchor{#2}{Vlab}, #4] at (#2-Vlab) {#3};
}
\tikzset{growing arrow/.style={decorate,
decoration={show path construction,
moveto code={},
lineto code={
\draw[line width=1pt,-{Stealth[width=12pt,length=12pt]}]
(\tikzinputsegmentfirst) --  (\tikzinputsegmentlast);
\fill ($ (\tikzinputsegmentlast)!6pt!0:(\tikzinputsegmentfirst) $) coordinate (aux)
($ (\tikzinputsegmentfirst)!0.5pt!90:(\tikzinputsegmentlast) $)
-- ($ (aux)!2pt!-90:(\tikzinputsegmentfirst) $)
--($ (aux)!2pt!90:(\tikzinputsegmentfirst) $)
-- ($ (\tikzinputsegmentfirst)!0.5pt!-90:(\tikzinputsegmentlast) $) ;
},
curveto code={},
closepath code={},
}}}
